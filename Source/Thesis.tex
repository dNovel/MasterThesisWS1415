\documentclass[pagesize, paper=a4, fontsize=12pt,titlepage=true, headings=small, headnosepline, abstractoff, liststotoc, nochapterprefix, plainheadsepline, twoside]{scrreprt}
\usepackage[a4paper, left=40mm, right=30mm, top=20mm, bottom=30mm]{geometry}
\usepackage[utf8]{inputenc}
\usepackage[ngerman]{babel}
\usepackage{amsmath}
\usepackage{amsfonts}
\usepackage{amssymb}
\usepackage{makeidx}
\usepackage{setspace}
\usepackage{color}
\usepackage{cite} % Paket fuer die Zitation
% \usepackage{natbib} % Erweitertes paket für Zitate.
%\usepackage{sourcesanspro}
\usepackage[T1]{fontenc}
\usepackage{lmodern}
% Bilder Settings
\usepackage{graphicx}
\usepackage [singlelinecheck=false] {caption}
\usepackage{subcaption}
\usepackage{url}
\usepackage{scrpage2}
\usepackage [singlelinecheck=false] {caption}
\usepackage{pdfpages}
%\usepackage{harvard}
\usepackage{natbib}

% Paket fuer das anzeigen von Sourcecode
\usepackage{listings}
% Setze die Programmiersprache auf CSharp
\lstset{language=[Sharp]C} 

% Festlegung Art der Zitierung -NatDin für Deutschland: Abkuerzung Autor + Jahr
\bibliographystyle{natdin}
%plain

% Festlegen der Sprache
\selectlanguage{ngerman}

% Settings fuer den Sourcecode START
\definecolor{mywhite}{rgb}{1,1,1}
\definecolor{mygreen}{rgb}{0,0.4,0}
\definecolor{mygray}{rgb}{0.5,0.5,0.5}
\definecolor{mykeywordgray}{rgb}{0.2,0.2,0.2}
\definecolor{mymauve}{rgb}{0.58,0,0.82}
\definecolor{bggray}{rgb}{0.97,0.97,0.97}
\definecolor{titlegray}{rgb}{0.4,0.4,0.4}

% Farbe für die Überschriften
\addtokomafont{sectioning}{\color{titlegray}\rmfamily}

% URL Style
\urlstyle{rm}

\lstset{
backgroundcolor=\color{mywhite},  % choose the background color; you must add \usepackage{color} or \usepackage{xcolor}
basicstyle=\small, % the size of the fonts that are used for the code
breakatwhitespace=false,         % sets if automatic breaks should only happen at whitespace
breaklines=true,                 % sets automatic line breaking
captionpos=b,                    % sets the caption-position to bottom
commentstyle=\small\color{black},    % comment style
deletekeywords={...},            % if you want to delete keywords from the given language
escapeinside={\%*}{*)},          % if you want to add LaTeX within your code
extendedchars=true,              % lets you use non-ASCII characters; for 8-bits encodings only, does not work with UTF-8
frame=single,                    % adds a frame around the code
keepspaces=true,                 % keeps spaces in text, useful for keeping indentation of code (possibly needs columns=flexible)
keywordstyle=\color{mykeywordgray}\bfseries,       % keyword style
language=[Sharp]C,                 % the language of the code
morekeywords={*,Select,where,select,Write, from, in, orderby, IEnumerable, Where, OrderBy, FindIndex, List, Count, Insert, Remove},            % if you want to add more keywords to the set
numbers=left,                    % where to put the line-numbers; possible values are (none, left, right)
numbersep=10pt,                   % how far the line-numbers are from the code
numberstyle=\color{mykeywordgray}, % the style that is used for the line-numbers
rulecolor=\color{titlegray},         % if not set, the frame-color may be changed on line-breaks within not-black text (e.g. comments (green here))
showspaces=false,                % show spaces everywhere adding particular underscores; it overrides 'showstringspaces'
showstringspaces=false,          % underline spaces within strings only
showtabs=false,                  % show tabs within strings adding particular underscores
stepnumber=1,                    % the step between two line-numbers. If it's 1, each line will be numbered
stringstyle=\color{black},     % string literal style
tabsize=2,                       % sets default tabsize to 2 spaces
title=\lstname,                   % show the filename of files included with \lstinputlisting; also try caption instead of title
captionpos=t,
aboveskip=1\baselineskip,		% Platz über dem quellcode block
belowskip=1\baselineskip,			% Platz unter dem quellcode block
%morecomment=[il]{///}
}
% Settings fuer den Sourcecode ENDE

% Listings
\renewcommand{\lstlistlistingname}{Verzeichnis der Sourcecode Beispiele}
\renewcommand{\lstlistingname}{Sourcecode Beispiele}

% Autoren
\author{
Dominik Steffen \and
Erstbetreuer: Prof. Christoph Müller, Fakultät DM \and
Zweitbetreuer: Prof. Dr. Wolfgang Taube, Fakultät DM
}


% Titel
\title{Splitting Game Development Processes for Good}
\subtitle{Konzeption und Implementierung eines Beispielhaften Game Authoring Prozesses unter betrachtung von Game Engine Tool Development Aspekten .... TBD}
\parindent 0pt


%%%%%%%%%%%%%%%%%%%%%%%%%%%%%%%%%%%%%%%%%%%%%%%%%%%%%%%%%%%%%%%%%%%%%%%%%%%%%%%%
%	Commands START - Makros
%%%%%%%%%%%%%%%%%%%%%%%%%%%%%%%%%%%%%%%%%%%%%%%%%%%%%%%%%%%%%%%%%%%%%%%%%%%%%%%%
% C# makro OHNE space nach dem logo
\newcommand{\CS}{C\texttt{\#}}
% C# makro MIT space nach dem logo
\newcommand{\CSS}{C\texttt{\# }}
% C++ Logo
\newcommand{\CPP}{C\nolinebreak\hspace{-.05em}\raisebox{.4ex}{\tiny\bf +}\nolinebreak\hspace{-.10em}\raisebox{.4ex}{\tiny\bf +}}
% LINQ For Geometry
\newcommand{\LFG}{LINQ For Geometry}
% LINQ For Geometry mit Space
\newcommand{\LFGS}{LINQ For Geometry }
% LINQ mit spaces links und rechts
\newcommand{\LQ}{ LINQ }
% Generic zeichen <T>
\newcommand{\GT}{\textless T\textgreater}
\newcommand{\GTS}{\textless T\textgreater\space}
% Lambda Zeichen in C#
\newcommand{\LAM}{ =\textgreater\space}
% HES
\newcommand{\HES}{Half-Edge Datenstruktur }
%%%%%%%%%%%%%%%%%%%%%%%%%%%%%%%%%%%%%%%%%%%%%%%%%%%%%%%%%%%%%%%%%%%%%%%%%%%%%%%%
%	Commands ENDE
%%%%%%%%%%%%%%%%%%%%%%%%%%%%%%%%%%%%%%%%%%%%%%%%%%%%%%%%%%%%%%%%%%%%%%%%%%%%%%%%


%%%%%%%%%%%%%%%%%%%%%%%%%%%%%%%%%%%%%%%%%%%%%%%%%%%%%%%%%%%%%%%%%%%%%%%%%%%%%%%%
%	Unterstrichene Kapitelüberschriften START
%%%%%%%%%%%%%%%%%%%%%%%%%%%%%%%%%%%%%%%%%%%%%%%%%%%%%%%%%%%%%%%%%%%%%%%%%%%%%%%%
\newcommand*{\ORIGchapterheadendvskip}{}%
\let\ORIGchapterheadendvskip=\chapterheadendvskip
\renewcommand*{\chapterheadendvskip}{%
\ORIGchapterheadendvskip
{%
\setlength{\parskip}{0pt}%
\noindent\rule[3\baselineskip]{\linewidth}{1pt}\par
}%
}
%%%%%%%%%%%%%%%%%%%%%%%%%%%%%%%%%%%%%%%%%%%%%%%%%%%%%%%%%%%%%%%%%%%%%%%%%%%%%%%%
%	Unterstrichene Kapitelüberschriften ENDE
%%%%%%%%%%%%%%%%%%%%%%%%%%%%%%%%%%%%%%%%%%%%%%%%%%%%%%%%%%%%%%%%%%%%%%%%%%%%%%%%

\newpage

\makeindex
\onehalfspacing
%\setuptoc{toc}{numbered}

\begin{document}
% Titelblatt START
%\maketitle
%\addcontentsline{toc}{chapter}{Titelblatt}
\includepdf[pages={1}]{Includes/deckblatt.pdf}
% Titelblatt ENDE


%%%%%%%%%%%%%%%%%%%%%%%%%%%%%%%%%%%%%%%%%%%%%%%%%%%%%%%%%%%%%%%%%%%%%%%%%%%%%%%%
%	Abstract START
%%%%%%%%%%%%%%%%%%%%%%%%%%%%%%%%%%%%%%%%%%%%%%%%%%%%%%%%%%%%%%%%%%%%%%%%%%%%%%%%
\newpage
\thispagestyle{empty}
\mbox{}

\begingroup
\newpage
\pagestyle{empty}
\renewcommand*{\chapterpagestyle}{empty}
\chapter*{Abstract}%
%\addcontentsline{toc}{chapter}{Abstract}
Arbeitsprozesse in heutigen Game Engines verlangen von Entwicklern meist das erlernen neuer Toolsets und das während eines meist sehr eingeschhränkten Projekt Zeitraums. Es wäre für Entwickler einfacher sich mit den bereits bekannten Tools zu beschäftigen und mit diesen großartige Ergebnisse zu erreichen. Designer müssen sich oft in unbekannte Editoren und SDKs einarbeiten während Entwickler sich in Grafische Editoren einarbeiten sollen um ihren Code an der richtigen Stelle des Projekts einzubinden.
Diese Arbeit baut eine Brücke zwischen beiden Welten. Durch die Konzeption und Umsetzung eines Software Tools und Entwicklungsprozesses wird eine Trennung der Abhängigkeiten in einem Projekt erreicht. Mit Hilfe eines Plugins ist es möglich, dass Designer oder Entwickler jederzeit mit ihren eigenen Tools in die Entwicklung eines Projektes einsteigen. Es wird ermöglicht mit Cinema 4D und einer IDE wie VS2013 an einem Projekt mit der FUSEE Engine zu arbeiten ohne die bereits bekannte Welt zu verlassen. Ein FUSEE Projektstruktur "managed" sich durch die Nutzung des entstandenen Cinema 4D Plugins und den generierten Visual Studio Solution Dateien selbst.
Das zuerst konzeptionell entworfene Tool wurde während dieser Arbeit umgesetzt und bietet ausreichende Basisfunktionalität um ein Projekt als Entwickler als auch als Artist zu erstellen und zu bearbeiten. Hierzu wurden verschiedene Konzepte betrachetet und andere GameEngines auf Workflow und Anwendbarkeit untersucht. Es wurden einige Kernkonzepte erkannt und für eine Implementierung in FUSEE Uniplug analysiert und weiter entwickelt.
\clearpage
\endgroup
%%%%%%%%%%%%%%%%%%%%%%%%%%%%%%%%%%%%%%%%%%%%%%%%%%%%%%%%%%%%%%%%%%%%%%%%%%%%%%%%
%	Abstract ENDE
%%%%%%%%%%%%%%%%%%%%%%%%%%%%%%%%%%%%%%%%%%%%%%%%%%%%%%%%%%%%%%%%%%%%%%%%%%%%%%%%

%%%%%%%%%%%%%%%%%%%%%%%%%%%%%%%%%%%%%%%%%%%%%%%%%%%%%%%%%%%%%%%%%%%%%%%%%%%%%%%%
%	Versicherung START
%%%%%%%%%%%%%%%%%%%%%%%%%%%%%%%%%%%%%%%%%%%%%%%%%%%%%%%%%%%%%%%%%%%%%%%%%%%%%%%%
\newpage
\thispagestyle{empty}
\mbox{}

\begingroup
\pagestyle{empty}
\newpage
\renewcommand*{\chapterpagestyle}{empty}
\chapter*{Eidesstattliche Erkl"arung}%
%\addcontentsline{toc}{chapter}{Eidesstattliche Erkl"arung}
Ich erkläre hiermit an Eides statt, dass ich die vorliegende Masterthesis selbständig und ohne 
unzulässige fremde Hilfe angefertigt habe. Alle verwendeten Quellen und Hilfsmittel die sowohl zum schreiben dieser Arbeit als auch zum Entwickeln des dazugeh"origen Sourcecodes benutzt wurden, habe ich angegeben.

\vspace*{3cm}
\hspace*{\fill}\begin{tabular}{@{}l@{}}\hline
\makebox[9cm]{Dominik Steffen, K"ussaberg den \today}
\end{tabular}
\clearpage
\endgroup
%%%%%%%%%%%%%%%%%%%%%%%%%%%%%%%%%%%%%%%%%%%%%%%%%%%%%%%%%%%%%%%%%%%%%%%%%%%%%%%%
%	Versicherung ENDE
%%%%%%%%%%%%%%%%%%%%%%%%%%%%%%%%%%%%%%%%%%%%%%%%%%%%%%%%%%%%%%%%%%%%%%%%%%%%%%%%

%%%%%%%%%%%%%%%%%%%%%%%%%%%%%%%%%%%%%%%%%%%%%%%%%%%%%%%%%%%%%%%%%%%%%%%%%%%%%%%%
%	Logo START
%%%%%%%%%%%%%%%%%%%%%%%%%%%%%%%%%%%%%%%%%%%%%%%%%%%%%%%%%%%%%%%%%%%%%%%%%%%%%%%%
\newpage
\thispagestyle{empty}
\mbox{}

\begingroup
\newpage
\thispagestyle{empty}
\vspace*{8cm}
%\includegraphics[width=\linewidth]{Bilder/Logo}
\vspace*{1cm}
\begin{quote}
"Hier steht ein wichtiges Zitat zur Entstehung dieser Arbeit."
\end{quote} - TBD.
\vspace*{5cm}
\\Kontakt:\\
Dominik Steffen (Matr.-Nr.: 245857)\\
Hochschule Furtwangen\\
E-Mail:\\
dominik.steffen@hs-furtwangen.de, dominik.steffen@gmail.com\\
\endgroup
%%%%%%%%%%%%%%%%%%%%%%%%%%%%%%%%%%%%%%%%%%%%%%%%%%%%%%%%%%%%%%%%%%%%%%%%%%%%%%%%
%	Logo START
%%%%%%%%%%%%%%%%%%%%%%%%%%%%%%%%%%%%%%%%%%%%%%%%%%%%%%%%%%%%%%%%%%%%%%%%%%%%%%%%
\newpage
\thispagestyle{empty}
\mbox{}

% Inhaltsverzeichnis START
\begingroup
	\clearpage
	\pagestyle{empty}
	%\addcontentsline{toc}{chapter}{Inhaltsverzeichnis} 
	\tableofcontents
	\clearpage
\endgroup
% Inhaltsverzeichnis ENDE
\newpage
\thispagestyle{empty}
\mbox{}

%%%%%%%%%%%%%%%%%%%%%%%%%%%%%%%%%%%%%%%%%%%%%%%%%%%%%%%%%%%%%%%%%%%%%%%%%%%%%%%%
% Inhalt START
%%%%%%%%%%%%%%%%%%%%%%%%%%%%%%%%%%%%%%%%%%%%%%%%%%%%%%%%%%%%%%%%%%%%%%%%%%%%%%%%

% Passe Seitenzahlen wieder an START
\renewcommand*{\chapterpagestyle}{plain}
\pagestyle{plain}
\setcounter{page}{0}
% Passe Seitenzahlen wieder an ENDE

%%%%%%
%	Einführung / Einleitung START
%%%%%%
\chapter{Anforderungen, Ziele und eine Fragestellung}

\section{Anforderung}
\section{Ziele}
\section{Fragestellung}

\section{Entwicklungsprozesse in Interaktiver 3D Software und Games}
\subsection{Projektmanagement Modelle}
% Warhorse Scrum, Waterfall, etc.
Um große Projekte wie Computergames oder Interaktive Software zu entwickeln, bedarf es meist einer detaillierten Planung und einer exakten Rollenverteilung im Entwicklerteam. Es existieren verschiedene Methoden des Projektmanagement auf welche hier kurz im Zusammenhang mit der Arbeit eingegangen werden soll. Einige der Projektmanagement Modelle wirken auf die Arbeitsweise der Teammitglieder aus. Daher wird diese Arbeit hier keinen Umfassenden Überblick über Projektmangament Methoden geben, sondern nur solche Ansprechen die sich direkt oder indirekt stark auf das Tool Development auswirken.
\subsubsection{Agile Modelle vs. klassische Modelle}
Viele Entwickler (Ubisoft, siehe \citet{MKG:Schmitz2014}) setzen heute auf moderne Modelle zum Entwickeln von Software. Die so genannten agilen Modelle (wie Beispielsweise Scrum,  Extreme Programming und Feature Driven Development) ermöglichen meist das schnelle (agile) reagieren auf plötzlich auftauchende schwierige Situationen. Klassische Modelle (Wasserfallmodell, Spiralmodell) haben hier meist Probleme durch ungleich höhere Bürokratie und Komplexität und benötigen ein Zeitaufwändigeres re-iterieren im Falle von Updates und Umstrukturierungen in Folge von unvorhergesehenen Ereignissen und Problemen. 
\subsubsection{Scrum}
Der Scrum Prozess tauchte das erste mal in der Veröffentlichung “The New New Product Development Game” von Hirotaka Takeuchi and Ikujiro Nonaka 1986 auf - damals nicht unbedingt in der Software- sondern der allgemeinen Produktentwicklung eingesetzt. Seitdem hat sich das Modell weiter entwickelt und erfreut sich bei innovativen Softareprojekten im Games und Indie-Games Bereich (auch und meist wohl auch vor allem im Tool Development) sehr großer Beliebtheit. Die Entwickler CCP und Warhorse Studios hatten hierzu eigene Videos und Artikel veröffentlicht, siehe \citet{CCP:ScrumAndAgile2009}, \citet{WH:Scrum_2013}, \citet{WH:Scrum_Video_2013}.

Ein Scrum Entwicklerteam ist mit folgenden Rollen besetzt:
\begin{itemize}
\item Product Owner
\item Entwicklungsteam
\item Scrum Master
\end{itemize}

Bei diesen Rollen handelt es sich um das interne Scrum Team - das Entwicklungsteam des Produktes. Scrum kann innerhalb eines Projektes und Teams beliebig heruntergebrochen werden, bis die gewünschte größe eines Entwicklerteams erreicht wird. Externe Rollen wie Stakeholder etc. verlagern sich somit auf andere interne Projektleiter oder Teammitglieder. Aus diesem Grund ist das Model gut für die Entwicklung von Development Tools und Toolkits geeignet. Mit Hilfe des Models, können benötigte Toolkits während einer Projektlaufzeit schnell und effizient entwickelt werden ohne dass ein schwerfälliger Bürokratischer Prozess die Entwicklung blockiert. Somit ergänzt sich dieser Prozess gut mit dem doch eher agilen entwickeln von Developement Tools während der Projektlaufzeit - denn in den seltensten Fällen wurde vor dem Beginn des Projekts daran gedacht alle nötigen Tools bereitzustellen. Oftmals ergeben sich auch während der Entwicklung neue Herausforderungen für das Team welche nach neuen Tools verlangen.

\subsection{Internes Tool Developing anstatt Tool licencing} % Titel verbessern
Internes Tool Development ist ein wichtiger Aspekt im Team eines Games und Software Entwicklerteams. Erich Bethke berichtet in Game Development and Production davon, dass Michael Abrash ihm einst mitteilte, “dass 50\% der Entwickler Arbeit bei idSoftware in das Tool Development fliesse.” Vgl. \citet[Seite 44]{Bethke2003}. Nun ist das bereits eine Weile her, allerdings hat sich an der Relevanz des Themas kaum etwas getan. Sony hat für den Release der Playstation 4 ein Development Kit \footnote{Interview mit dem SNSystems Team \citet{DVLP:Freeman_2014}} für die internen Entwickler Studios  erstellen lassen, welches bereits während der Planung und Entwicklung der Konsole entwickelt wurde. Sony hat diese Prozedur perfektioniert und lässt die eigenen Tools sogar in einem eigens dafür gegründeten Unternehmen für die eigenen Studios erstellen \footnote{SNSystems \url{http://www.snsystems.com/}}. Sony hat im Herbst 2014 den für Playstation 3 Spiele eigens entwickelten Welt Editor “Level Editor” als Open Source Software veröffentlicht \footnote{\cite{GS:Sony_LE_2014}}

\section{Mitglieder eines Entwicklerteams}
Hier gibt diese Arbeit einen kurzen Überblick über die gängigsten Mitglieder eines Entwicklerteams. Grob können Mitglieder in die folgenden drei Gruppen aufgeteilt werden - Artists, Designer, Engineer. Jede Gruppe arbeitet hierbei jedoch interdisziplinär mit den anderen zusammen, kümmert sich aber doch um die ganz eigenen Bestandteile eines Produktes. Es ist jedoch durchaus so, dass jede Gruppe ihre eigenen Tools und Methoden verwendet. Dieser Ansatz wird in der Konzeptionierung dieser Arbeit aufgegriffen und weiter verfolgt.

\subsection{Artists}
% Animator
% World Builders
Artists sind in einem Games Projekt für jegliche repräsentation der Spiellogik nach Außen zuständig. Hierbei handelt es sich um Modelle, Texturen, User Interfaces und weitere Oberflächen.
\subsection{Designer}
Designer (Gamedesigner) arbeiten eng mit Artists und Engineers zusammen. Sie schreiben oft Skripte und kleine Implementierungen oder verbessern Grafiken oder Spielfunktionen. Sie verwenden Assets aus der Designabteilung und fügen diese mit Skripten zusammen.
% Level
% Scripter
% UI
\subsection{Engineer}
Engineers / Ingenieure arbeiten meist am Kern der Applikation und schrieben den Source Code für die Anwendung, Engine, Netzwerkfunktionen, KI, und Tools. Diese Entwickler arbeiten hauptsächlich in einer IDE \footnote{Integrated Developement Environment} wie Visual Studio (auf welches sich das zu dieser Arbeit konzeptionierte Tool bezieht) oder XCode \footnote{X-Code ist nur für MacOSX erhältlich}. Der in der IDE geschriebene Code wird dann von den Engineers selbst oder von Game Designer in der Engine verwendet. Hierbei kann sich das Tätigkeitsfeld ausweiten bis hin zur Entwicklung von Gamelogic \footnote{Logik des Spiels, ermöglicht das interagieren etc. mit und in der Software}.
% Tools
Diese Arbeit bezieht sich auf den Bereich des Tool Development. Hierbei entwickelt ein kleines Team - meist während oder vor der eigentlichen Arbeit an einem Projekt die Tools für die restlichen Entwickler des Projektes. Diese Tool Palette kann von Textureditoren bis hin zu kompletten Welteditoren fast alles vorstellbare enthalten. Verschiedene Studios haben eigene Tool Developer Teams, welche sich nur um diesen Bereich des Produktes kümmern. Diese Teams betreuen auch meist den Modding Support für ein fertiges veröffentlichtes Produkt. Beispiele für Modding Tools sind z.B. das RedKit von CDProject Red für das Spiel The Witcher 1 und 2, der LevelEditor von Sony der in einer Open Source Version vorliegt oder das Creation Kit von Bethesda Softworks welches einen Modding Support für die Spiele der The Elder Scrolls Reihe bereit stellt.
% Graphics
% Network
% AI
% Sound
\subsection{Weitere für diese Arbeit nicht relevante}


\section{Stakeholderanalyse intern}
Das bedeutet, diese Gruppen müssen bei der entwicklung eines Tools beachtet werden. Am besten wäre eine Rücksprache mit diesen Teammitgliedern und anschliessend eine analyse der Bedürfnisse für die jeweiligen Aufgabengebiete. Hierzu würde es sich empfehlen ein Gespräch zu führen und weiterhin einen praktischen Besuch am Arbeitsplatz zu vereinbaren.

\section{Ein Arbeitsprozess wird entwickelt}
\subsection{Game Authoring / Game Development}
Was ist das, was beschreibt es, wieso ist es hier relevant?
\subsection{Tool Development}
% Nach Wihlidal ... Warum wurde oben geklärt, hier den Prozess erläutern.
% Möglichkeiten vorstellen.
% Beispiele geben.

\subsection{Planung}
\begin{itemize}
\item Eine Software soll es ermöglichen, dass Artists, Designer und Developer an ein und dem selben Projekt arbeiten können ohne die gewohnte Arbeitsumgebung (3D-Modellierungssoftware, IDE) zu verlassen und etwas komplett neues (Level-Editor) zu erlernen.
\item Die komplette Entwicklung und der Entwurf bedienen sich der FUSEE Engine.
\item Ziel ist es in Cinema 4D ein FUSEE Projekt anzulegen, zu speichern und es zu öffnen (In Form einer Szene, Level, Welt.). Weiterhin müssen "Assets" ins Spiel integriert werden können die von Artists, Designern und Entwicklern bearbeitet werden können.
\item Das Projekt sollte aus C4D heraus gebaut werden können.
\end{itemize}
%Hier könnte man einen Hinweis auf die Entwicklungsmethoden von id Software geben. Möglicherweise wäre eine Erwähnung der Pre-Rage Zeiten sinnvoll.

\subsection{Analyse}
\begin{itemize}
\item Eine Stakeholderanalyse schafft klarheit, welche Parteien mit dem zu erstellenden Tool arbeiten müssen.
\item Die Game Engines sollten Kategorisiert werden. Hierzu ist ein "Gitternetz" aufzustellen.
\item Verschiedene Game Engines müssen darauf untersucht werden, wie die Arbeitsschritte in diesen zu erledigen sind. Jede Untersuchung sollte dokumentiert werden.
\item Es ist zuerst zu analysieren, welche Schritte für welche Art der Arbeit notwendig sind. Hierzu werden Usecases der verschiedenen Rollen und Aufgaben erstellt.
\end{itemize}

\subsection{Design}
\begin{itemize}
\item Aufgrund der erstellten Use Cases wird ein System Design erstellt.
\item Das Systemdesing sollte Roundtrips vermeiden
\item Hierzu können Partial Classes untersucht werden.
\item Es soll möglichst wenig "geparsed" oder "konvertiert" werden
\item Dateien sollen Version Control kompatibel bleiben (wenig bis keine Binarys)
\item Das Projekt muss Strukturiert sein
\item Eine Fusee Solution soll gehandelt werden können
\end{itemize}

\subsection{Implementierung}
\begin{itemize}
\item Die Software wird auf basis der FUSEE Engine und Cinema4D R16 erstellt.
\end{itemize}

\subsubsection{Asset Pipeline und Feedback}
SEHR KRITISCHE MARKE! Das Asset aus dem Editor in die Engine bekommen, die Darstellung etc kontrollieren.
Zeitkritisch beim export etc.
Carter Seite 6 und folgende.
%Begriffserklärung eines Asset, genaue Definitionen anmerken.

\subsubsection{Die Struktur von Game Assets}
Assets und das aufbrechen in Bestandteile eines Projektes. Level etc. bestehen aus Assets und mehr.
\subsection{Warum eine Trennung von Code und Content?}
%%%%%%
%	Einführung / Einleitung ENDE
%%%%%%

%%%%%%
%	Hauptteil START
%%%%%%
\chapter{Entwicklung eines Konzeptes}

% Planung
\section{Use Cases der verschiedenen Entwickler}
\subsection{Was möchten Artists?}
\subsection{Was möchten Designer?}
\subsection{Was möchten Entwickler?}

% Analyse
\subsection{Projekt bezogen}
\subsubsection{Projekt anlegen}
\subsubsection{Projekt öffnen}
\subsubsection{Projekt speichern}
\subsubsection{Projekt bauen}
\subsubsection{In Projekt einsteigen}
\subsubsection{Projekt clonen etc.}

\subsection{Prozess bezogen}
\subsubsection{Gleichzeitig an Projekt arbeiten}
\subsubsection{Model Datei importieren}
\subsubsection{Gleichzeitig an einem Objekt arbeiten}

\section{Aktuelle Engines und deren Arbeitsprozesse}
\subsection{Prozesse in Game Engines und einem Framework}
\subsection{Unreal Engine 4}
\subsection{Unity 3D}
\subsection{idTech X}
\subsection{Weitere}

% Design
\section{Konzeptentwurf}
\subsection{Systemdesign für ein Plugin}
\subsection{Systemdesign für einen Project-Handler}
\subsection{Entfernen von Abhängigkeiten}
\subsection{Zeitersparnis durch bekannte Tools}
\subsection{Warum Fusee und Cinema 4D?}

% Implementierung
\section{Die Implementierung}
\subsection{Cinema 4D Plugin API und SDK}
\subsubsection{Uniplug}
\subsection{Fusee}
\subsubsection{Der Fusee Szenengraph}
Das Fusee Level, Welt wie auch immer es hier bezeichnet werden sollte. Eine Basis wird gebraucht. Hierzu eine Zentrale anlaufstelle, ein Spiele “Kernel”? Irgend etwas dass im Zentrum steht.
Alles andere muss auf Level etc aufgeteilt werden und bis zum einzelnen Asset heruntergebrochen werden.

\section{Das eigentliche Plugin}
\subsection{Visualisierung der Systemarchitektur}
\subsubsection{Welche Programme und Systeme sind beteiligt?}
\subsection{Generieren eines Fusee Projektes}
\subsection{Code Generation und die Vermeidung von Roundtrips (nicht so ganz roundtrips, generierung um generierung etc.)}
\subsection{XPresso Schaltungen - Programmieren ohne Programmieren}
\subsection{Partial Classes in .NET}

%%%%%%
%	Hauptteil ENDE
%%%%%%


%%%%%%
%	Schluss START
%%%%%%
\chapter{Ergebnisse und Erkentnisse}
\section{Game Authoring Entwicklungsprozesse jetzt und in Zukunft}
\section{Wie weit ist die Implementierung fortgeschritten?}
\section{Welcher Mehrwert wurde erreicht?}
\section{Integration des Systems in den weiteren Projektverlauf von FUSEE}
%%%%%%
%	Schluss ENDE
%%%%%%


%%%%%%%%%%%%%%%%%%%%%%%%%%%%%%%%%%%%%%%%%%%%%%%%%%%%%%%%%%%%%%%%%%%%%%%%%%%%%%%%
% Inhalt ENDE
%%%%%%%%%%%%%%%%%%%%%%%%%%%%%%%%%%%%%%%%%%%%%%%%%%%%%%%%%%%%%%%%%%%%%%%%%%%%%%%%
\part*{Anhang}


%%%%%%%%%%%%%%%%%%%%%%%%%%%%%%%%%%%%%%%%%%%%%%%%%%%%%%%%%%%%%%%%%%%%%%%%%%%%%%%%
% Source Code Verzeichnis START
%%%%%%%%%%%%%%%%%%%%%%%%%%%%%%%%%%%%%%%%%%%%%%%%%%%%%%%%%%%%%%%%%%%%%%%%%%%%%%%%
\lstlistoflistings
%%%%%%%%%%%%%%%%%%%%%%%%%%%%%%%%%%%%%%%%%%%%%%%%%%%%%%%%%%%%%%%%%%%%%%%%%%%%%%%%
% Source Code Verzeichnis ENDE
%%%%%%%%%%%%%%%%%%%%%%%%%%%%%%%%%%%%%%%%%%%%%%%%%%%%%%%%%%%%%%%%%%%%%%%%%%%%%%%%

%%%%%%%%%%%%%%%%%%%%%%%%%%%%%%%%%%%%%%%%%%%%%%%%%%%%%%%%%%%%%%%%%%%%%%%%%%%%%%%%
% Tabellen Verzeichnis START
%%%%%%%%%%%%%%%%%%%%%%%%%%%%%%%%%%%%%%%%%%%%%%%%%%%%%%%%%%%%%%%%%%%%%%%%%%%%%%%%
\listoftables
%%%%%%%%%%%%%%%%%%%%%%%%%%%%%%%%%%%%%%%%%%%%%%%%%%%%%%%%%%%%%%%%%%%%%%%%%%%%%%%%
% Tabellen Verzeichnis ENDE
%%%%%%%%%%%%%%%%%%%%%%%%%%%%%%%%%%%%%%%%%%%%%%%%%%%%%%%%%%%%%%%%%%%%%%%%%%%%%%%%

%%%%%%%%%%%%%%%%%%%%%%%%%%%%%%%%%%%%%%%%%%%%%%%%%%%%%%%%%%%%%%%%%%%%%%%%%%%%%%%%
% Abbildungsverzeichnis START
%%%%%%%%%%%%%%%%%%%%%%%%%%%%%%%%%%%%%%%%%%%%%%%%%%%%%%%%%%%%%%%%%%%%%%%%%%%%%%%%
\listoffigures
%%%%%%%%%%%%%%%%%%%%%%%%%%%%%%%%%%%%%%%%%%%%%%%%%%%%%%%%%%%%%%%%%%%%%%%%%%%%%%%%
% Abbildungsverzeichnis ENDE
%%%%%%%%%%%%%%%%%%%%%%%%%%%%%%%%%%%%%%%%%%%%%%%%%%%%%%%%%%%%%%%%%%%%%%%%%%%%%%%%

%%%%%%%%%%%%%%%%%%%%%%%%%%%%%%%%%%%%%%%%%%%%%%%%%%%%%%%%%%%%%%%%%%%%%%%%%%%%%%%%
% Bilbiographie START
%%%%%%%%%%%%%%%%%%%%%%%%%%%%%%%%%%%%%%%%%%%%%%%%%%%%%%%%%%%%%%%%%%%%%%%%%%%%%%%%
\nocite{*}
\bibliography{Bibtex/VerzeichnisBuecher}
\bibliographystyle{natdin}
\addcontentsline{toc}{chapter}{Literaturverzeichnis}
\newpage
%%%%%%%%%%%%%%%%%%%%%%%%%%%%%%%%%%%%%%%%%%%%%%%%%%%%%%%%%%%%%%%%%%%%%%%%%%%%%%%%
% Bilbiographie ENDE
%%%%%%%%%%%%%%%%%%%%%%%%%%%%%%%%%%%%%%%%%%%%%%%%%%%%%%%%%%%%%%%%%%%%%%%%%%%%%%%%

%%%%%%%%%%%%%%%%%%%%%%%%%%%%%%%%%%%%%%%%%%%%%%%%%%%%%%%%%%%%%%%%%%%%%%%%%%%%%%%%
% UML START
%%%%%%%%%%%%%%%%%%%%%%%%%%%%%%%%%%%%%%%%%%%%%%%%%%%%%%%%%%%%%%%%%%%%%%%%%%%%%%%%
\chapter*{UML Diagramme}
\addcontentsline{toc}{chapter}{UML Diagramme}
%%%%%%%%%%%%%%%%%%%%%%%%%%%%%%%%%%%%%%%%%%%%%%%%%%%%%%%%%%%%%%%%%%%%%%%%%%%%%%%%
% UML ENDE
%%%%%%%%%%%%%%%%%%%%%%%%%%%%%%%%%%%%%%%%%%%%%%%%%%%%%%%%%%%%%%%%%%%%%%%%%%%%%%%%
\end{document}
