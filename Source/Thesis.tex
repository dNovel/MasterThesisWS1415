% !!!SUCHE NACH IMPORTANT ZUM CHECKEN!!!

\documentclass[pagesize, paper=a4, fontsize=12pt,titlepage=true, headings=small, headnosepline, abstractoff, liststotoc, nochapterprefix, plainheadsepline, twoside]{scrreprt}
\usepackage[a4paper, left=40mm, right=30mm, top=20mm, bottom=30mm]{geometry}
\usepackage[utf8]{inputenc}
\usepackage[ngerman]{babel}
\usepackage{amsmath}
\usepackage{amsfonts}
\usepackage{amssymb}
\usepackage{makeidx}
\usepackage{setspace}
\usepackage{color}
\usepackage{cite} % Paket fuer die Zitation
% \usepackage{natbib} % Erweitertes paket für Zitate.
%\usepackage{sourcesanspro}
\usepackage[T1]{fontenc}
\usepackage{lmodern}
% Bilder Settings
\usepackage{graphicx}
\usepackage [singlelinecheck=false] {caption}
\usepackage{subcaption}
\usepackage{url}
\usepackage{scrpage2}
\usepackage [singlelinecheck=false] {caption}
\usepackage{pdfpages}

% Paket fuer das anzeigen von Sourcecode
\usepackage{listings}
% Setze die Programmiersprache auf CSharp
\lstset{language=[Sharp]C} 

% Festlegung Art der Zitierung - Havardmethode: Abkuerzung Autor + Jahr
\bibliographystyle{plain}
%plain

% Festlegen der Sprache
\selectlanguage{ngerman}

% Settings fuer den Sourcecode START
\definecolor{mywhite}{rgb}{1,1,1}
\definecolor{mygreen}{rgb}{0,0.4,0}
\definecolor{mygray}{rgb}{0.5,0.5,0.5}
\definecolor{mykeywordgray}{rgb}{0.2,0.2,0.2}
\definecolor{mymauve}{rgb}{0.58,0,0.82}
\definecolor{bggray}{rgb}{0.97,0.97,0.97}
\definecolor{titlegray}{rgb}{0.4,0.4,0.4}

% Farbe für die Überschriften
\addtokomafont{sectioning}{\color{titlegray}\rmfamily}


\lstset{
backgroundcolor=\color{mywhite},  % choose the background color; you must add \usepackage{color} or \usepackage{xcolor}
basicstyle=\small, % the size of the fonts that are used for the code
breakatwhitespace=false,         % sets if automatic breaks should only happen at whitespace
breaklines=true,                 % sets automatic line breaking
captionpos=b,                    % sets the caption-position to bottom
commentstyle=\small\color{black},    % comment style
deletekeywords={...},            % if you want to delete keywords from the given language
escapeinside={\%*}{*)},          % if you want to add LaTeX within your code
extendedchars=true,              % lets you use non-ASCII characters; for 8-bits encodings only, does not work with UTF-8
frame=single,                    % adds a frame around the code
keepspaces=true,                 % keeps spaces in text, useful for keeping indentation of code (possibly needs columns=flexible)
keywordstyle=\color{mykeywordgray}\bfseries,       % keyword style
language=[Sharp]C,                 % the language of the code
morekeywords={*,Select,where,select,Write, from, in, orderby, IEnumerable, Where, OrderBy, FindIndex, List, Count, Insert, Remove},            % if you want to add more keywords to the set
numbers=left,                    % where to put the line-numbers; possible values are (none, left, right)
numbersep=10pt,                   % how far the line-numbers are from the code
numberstyle=\color{mykeywordgray}, % the style that is used for the line-numbers
rulecolor=\color{titlegray},         % if not set, the frame-color may be changed on line-breaks within not-black text (e.g. comments (green here))
showspaces=false,                % show spaces everywhere adding particular underscores; it overrides 'showstringspaces'
showstringspaces=false,          % underline spaces within strings only
showtabs=false,                  % show tabs within strings adding particular underscores
stepnumber=1,                    % the step between two line-numbers. If it's 1, each line will be numbered
stringstyle=\color{black},     % string literal style
tabsize=2,                       % sets default tabsize to 2 spaces
title=\lstname,                   % show the filename of files included with \lstinputlisting; also try caption instead of title
captionpos=t,
aboveskip=1\baselineskip,		% Platz über dem quellcode block
belowskip=1\baselineskip,			% Platz unter dem quellcode block
%morecomment=[il]{///}
}
% Settings fuer den Sourcecode ENDE

% Listings
\renewcommand{\lstlistlistingname}{Verzeichnis der Sourcecode Beispiele}
\renewcommand{\lstlistingname}{Sourcecode Beispiele}

% Autoren
\author{
Dominik Steffen \and
Erstbetreuer: Prof. Christoph Müller, Fakultät DM \and
Zweitbetreuer: Prof. Dr. Wolfgang Taube, Fakultät DM
}


% Titel
\title{Splitting Game Development Processes for Good}
\subtitle{Konzeption und Implementierung eines Beispielhaften Game Authoring Prozesses unter betrachtung von Game Engine Tool Development Aspekten .... TBD}
\parindent 0pt


%%%%%%%%%%%%%%%%%%%%%%%%%%%%%%%%%%%%%%%%%%%%%%%%%%%%%%%%%%%%%%%%%%%%%%%%%%%%%%%%
%	Commands START - Makros
%%%%%%%%%%%%%%%%%%%%%%%%%%%%%%%%%%%%%%%%%%%%%%%%%%%%%%%%%%%%%%%%%%%%%%%%%%%%%%%%
% C# makro OHNE space nach dem logo
\newcommand{\CS}{C\texttt{\#}}
% C# makro MIT space nach dem logo
\newcommand{\CSS}{C\texttt{\# }}
% C++ Logo
\newcommand{\CPP}{C\nolinebreak\hspace{-.05em}\raisebox{.4ex}{\tiny\bf +}\nolinebreak\hspace{-.10em}\raisebox{.4ex}{\tiny\bf +}}
% LINQ For Geometry
\newcommand{\LFG}{LINQ For Geometry}
% LINQ For Geometry mit Space
\newcommand{\LFGS}{LINQ For Geometry }
% LINQ mit spaces links und rechts
\newcommand{\LQ}{ LINQ }
% Generic zeichen <T>
\newcommand{\GT}{\textless T\textgreater}
\newcommand{\GTS}{\textless T\textgreater\space}
% Lambda Zeichen in C#
\newcommand{\LAM}{ =\textgreater\space}
% HES
\newcommand{\HES}{Half-Edge Datenstruktur }
%%%%%%%%%%%%%%%%%%%%%%%%%%%%%%%%%%%%%%%%%%%%%%%%%%%%%%%%%%%%%%%%%%%%%%%%%%%%%%%%
%	Commands ENDE
%%%%%%%%%%%%%%%%%%%%%%%%%%%%%%%%%%%%%%%%%%%%%%%%%%%%%%%%%%%%%%%%%%%%%%%%%%%%%%%%


%%%%%%%%%%%%%%%%%%%%%%%%%%%%%%%%%%%%%%%%%%%%%%%%%%%%%%%%%%%%%%%%%%%%%%%%%%%%%%%%
%	Unterstrichene Kapitelüberschriften START
%%%%%%%%%%%%%%%%%%%%%%%%%%%%%%%%%%%%%%%%%%%%%%%%%%%%%%%%%%%%%%%%%%%%%%%%%%%%%%%%
\newcommand*{\ORIGchapterheadendvskip}{}%
\let\ORIGchapterheadendvskip=\chapterheadendvskip
\renewcommand*{\chapterheadendvskip}{%
\ORIGchapterheadendvskip
{%
\setlength{\parskip}{0pt}%
\noindent\rule[3\baselineskip]{\linewidth}{1pt}\par
}%
}
%%%%%%%%%%%%%%%%%%%%%%%%%%%%%%%%%%%%%%%%%%%%%%%%%%%%%%%%%%%%%%%%%%%%%%%%%%%%%%%%
%	Unterstrichene Kapitelüberschriften ENDE
%%%%%%%%%%%%%%%%%%%%%%%%%%%%%%%%%%%%%%%%%%%%%%%%%%%%%%%%%%%%%%%%%%%%%%%%%%%%%%%%

\newpage

\makeindex
\onehalfspacing
%\setuptoc{toc}{numbered}

\begin{document}
% Titelblatt START
%\maketitle
%\addcontentsline{toc}{chapter}{Titelblatt}
\includepdf[pages={1}]{Includes/deckblatt.pdf}
% Titelblatt ENDE


%%%%%%%%%%%%%%%%%%%%%%%%%%%%%%%%%%%%%%%%%%%%%%%%%%%%%%%%%%%%%%%%%%%%%%%%%%%%%%%%
%	Abstract START
%%%%%%%%%%%%%%%%%%%%%%%%%%%%%%%%%%%%%%%%%%%%%%%%%%%%%%%%%%%%%%%%%%%%%%%%%%%%%%%%
\newpage
\thispagestyle{empty}
\mbox{}

\begingroup
\newpage
\pagestyle{empty}
\renewcommand*{\chapterpagestyle}{empty}
\chapter*{Abstract}%
%\addcontentsline{toc}{chapter}{Abstract}
Arbeitsprozesse in heutigen Game Engines verlangen von Entwicklern meist das erlernen neuer Toolsets und das während eines meist zeitlich sehr beschränkten Projektes. Es wäre für Entwickler einfacher sich mit den bereits bekannten Tools zu beschäftigen und mit diesen großartige Ergebnisse zu erreichen. Designer müssen sich oft in unbekannte Editoren und SDKs einarbeiten während Entwickler sich in Grafische Editoren einarbeiten um ihren Code an der richtigen Stelle des Projekts zu platzieren.
Diese Arbeit baut eine Brücke zwischen beiden Welten. Durch die Konzeption und Umsetzung eines Software Tools wird eine Trennung der Abhängigkeiten in einem Projekt erreicht. Mit Hilfe eines Plugins ist es möglich, dass Designer oder Entwickler jederzeit mit ihren eigenen Tools in die Entwicklung eines Projektes einsteigen. Es wird ermöglicht mit Cinema 4D und einer IDE wie VS2013 an einem Projekt mit der FUSEE Engine zu arbeiten ohne die bereits bekannte Welt zu verlassen. Ein Engine Projektstruktur managed sich durch die Nutzung des entstandenen Plugins selbst.
Das zuerst Konzeptionell entworfene Tool wurde während dieser Arbeit umgesetzt und bietet ausreichende Basisfunktionalität um ein Projekt zu erstellen. Hierzu wurden verschiedene Konzepte betrachetet und andere Game Engines auf Workflow und Anwendbarkeit untersucht. Es wurden einige Kernkonzepte erkannt und für eine Implementierung in FUSEE analysiert und weiter entwickelt.
\clearpage
\endgroup
%%%%%%%%%%%%%%%%%%%%%%%%%%%%%%%%%%%%%%%%%%%%%%%%%%%%%%%%%%%%%%%%%%%%%%%%%%%%%%%%
%	Abstract ENDE
%%%%%%%%%%%%%%%%%%%%%%%%%%%%%%%%%%%%%%%%%%%%%%%%%%%%%%%%%%%%%%%%%%%%%%%%%%%%%%%%

%%%%%%%%%%%%%%%%%%%%%%%%%%%%%%%%%%%%%%%%%%%%%%%%%%%%%%%%%%%%%%%%%%%%%%%%%%%%%%%%
%	Versicherung START
%%%%%%%%%%%%%%%%%%%%%%%%%%%%%%%%%%%%%%%%%%%%%%%%%%%%%%%%%%%%%%%%%%%%%%%%%%%%%%%%
\newpage
\thispagestyle{empty}
\mbox{}

\begingroup
\pagestyle{empty}
\newpage
\renewcommand*{\chapterpagestyle}{empty}
\chapter*{Eidesstattliche Erkl"arung}%
%\addcontentsline{toc}{chapter}{Eidesstattliche Erkl"arung}
Ich erkläre hiermit an Eides statt, dass ich die vorliegende Masterthesis selbständig und ohne 
unzulässige fremde Hilfe angefertigt habe. Alle verwendeten Quellen und Hilfsmittel die sowohl zum schreiben dieser Arbeit als auch zum Entwickeln des dazugeh"origen Sourcecodes benutzt wurden, habe ich angegeben.

\vspace*{3cm}
\hspace*{\fill}\begin{tabular}{@{}l@{}}\hline
\makebox[9cm]{Dominik Steffen, K"ussaberg den \today}
\end{tabular}
\clearpage
\endgroup
%%%%%%%%%%%%%%%%%%%%%%%%%%%%%%%%%%%%%%%%%%%%%%%%%%%%%%%%%%%%%%%%%%%%%%%%%%%%%%%%
%	Versicherung ENDE
%%%%%%%%%%%%%%%%%%%%%%%%%%%%%%%%%%%%%%%%%%%%%%%%%%%%%%%%%%%%%%%%%%%%%%%%%%%%%%%%

%%%%%%%%%%%%%%%%%%%%%%%%%%%%%%%%%%%%%%%%%%%%%%%%%%%%%%%%%%%%%%%%%%%%%%%%%%%%%%%%
%	Logo START
%%%%%%%%%%%%%%%%%%%%%%%%%%%%%%%%%%%%%%%%%%%%%%%%%%%%%%%%%%%%%%%%%%%%%%%%%%%%%%%%
\newpage
\thispagestyle{empty}
\mbox{}

\begingroup
\newpage
\thispagestyle{empty}
\vspace*{8cm}
%\includegraphics[width=\linewidth]{Bilder/Logo}
\vspace*{1cm}
\begin{quote}
"Hier steht ein wichtiges Zitat zur Entstehung dieser Arbeit."
\end{quote} - TBD.
\vspace*{5cm}
\\Kontakt:\\
Dominik Steffen (Matr.-Nr.: 245857)\\
Hochschule Furtwangen\\
E-Mail:\\
dominik.steffen@hs-furtwangen.de, dominik.steffen@gmail.com\\
\endgroup
%%%%%%%%%%%%%%%%%%%%%%%%%%%%%%%%%%%%%%%%%%%%%%%%%%%%%%%%%%%%%%%%%%%%%%%%%%%%%%%%
%	Logo START
%%%%%%%%%%%%%%%%%%%%%%%%%%%%%%%%%%%%%%%%%%%%%%%%%%%%%%%%%%%%%%%%%%%%%%%%%%%%%%%%
\newpage
\thispagestyle{empty}
\mbox{}

% Inhaltsverzeichnis START
\begingroup
	\clearpage
	\pagestyle{empty}
	%\addcontentsline{toc}{chapter}{Inhaltsverzeichnis} 
	\tableofcontents
	\clearpage
\endgroup
% Inhaltsverzeichnis ENDE
\newpage
\thispagestyle{empty}
\mbox{}

%%%%%%%%%%%%%%%%%%%%%%%%%%%%%%%%%%%%%%%%%%%%%%%%%%%%%%%%%%%%%%%%%%%%%%%%%%%%%%%%
% Inhalt START
%%%%%%%%%%%%%%%%%%%%%%%%%%%%%%%%%%%%%%%%%%%%%%%%%%%%%%%%%%%%%%%%%%%%%%%%%%%%%%%%

% Passe Seitenzahlen wieder an START
\renewcommand*{\chapterpagestyle}{plain}
\pagestyle{plain}
\setcounter{page}{0}
% Passe Seitenzahlen wieder an ENDE

%%%%%%
%	Einführung / Einleitung START
%%%%%%
\chapter{Anforderungen, Ziele und eine Fragestellung}

\section{Entwicklungsprozesse in Games}
\subsection{Mitglieder eines Entwicklerteams}
\subsubsection{Designer}
\subsubsection{Entwickler}

\section{Ein Arbeitsprozess wird entwickelt}
\subsection{Game Authoring}
\subsection{Tool Development}
\subsection{Warum eine Trennung von Code und Content?}
%%%%%%
%	Einführung / Einleitung ENDE
%%%%%%

%%%%%%
%	Hauptteil START
%%%%%%
\chapter{Entwicklung eines Konzeptes}

\section{Use Cases der verschiedenen Entwickler}
\subsection{Was möchten Designer?}
\subsection{Was möchten Entwickler?}

\section{Aktuelle Engines und deren Arbeitsprozesse}
\subsection{Prozesse in Game Engines und einem Framework}
\subsection{Unreal Engine 4}
\subsection{Unity 3D}
\subsection{Android SDK}

\section{Konzeptentwurf}
\subsection{Entfernen von Abhängigkeiten}
\subsection{Zeitersparnis durch bekannte Tools}
\subsection{Warum Fusee und Cinema 4D?}

\section{Die Implementierung}
\subsection{Cinema 4D Plugin API und SDK}
\subsubsection{Uniplug}
\subsection{Fusee}
\subsubsection{Der Fusee Szenengraph}

\section{Das eigentliche Plugin}
\subsection{Generieren eines Fusee Projektes}
\subsection{Code Generation und die Vermeidung von Roundtrips (nicht so ganz roundtrips, generierung um generierung etc.)}
\subsection{XPresso Schaltungen - Programmieren ohne Programmieren}

%%%%%%
%	Hauptteil ENDE
%%%%%%


%%%%%%
%	Schluss START
%%%%%%
\chapter{Ergebnisse und Erkentnisse}
\section{Game Authoring Entwicklungsprozesse jetzt und in Zukunft}
\section{Wie weit ist die Implementierung fortgeschritten?}
\section{Welcher Mehrwert wurde erreicht?}
\section{Integration des Systems in den weiteren Projektverlauf von FUSEE}
%%%%%%
%	Schluss ENDE
%%%%%%


%%%%%%%%%%%%%%%%%%%%%%%%%%%%%%%%%%%%%%%%%%%%%%%%%%%%%%%%%%%%%%%%%%%%%%%%%%%%%%%%
% Inhalt ENDE
%%%%%%%%%%%%%%%%%%%%%%%%%%%%%%%%%%%%%%%%%%%%%%%%%%%%%%%%%%%%%%%%%%%%%%%%%%%%%%%%
\part*{Anhang}


%%%%%%%%%%%%%%%%%%%%%%%%%%%%%%%%%%%%%%%%%%%%%%%%%%%%%%%%%%%%%%%%%%%%%%%%%%%%%%%%
% Source Code Verzeichnis START
%%%%%%%%%%%%%%%%%%%%%%%%%%%%%%%%%%%%%%%%%%%%%%%%%%%%%%%%%%%%%%%%%%%%%%%%%%%%%%%%
\lstlistoflistings
%%%%%%%%%%%%%%%%%%%%%%%%%%%%%%%%%%%%%%%%%%%%%%%%%%%%%%%%%%%%%%%%%%%%%%%%%%%%%%%%
% Source Code Verzeichnis ENDE
%%%%%%%%%%%%%%%%%%%%%%%%%%%%%%%%%%%%%%%%%%%%%%%%%%%%%%%%%%%%%%%%%%%%%%%%%%%%%%%%

%%%%%%%%%%%%%%%%%%%%%%%%%%%%%%%%%%%%%%%%%%%%%%%%%%%%%%%%%%%%%%%%%%%%%%%%%%%%%%%%
% Tabellen Verzeichnis START
%%%%%%%%%%%%%%%%%%%%%%%%%%%%%%%%%%%%%%%%%%%%%%%%%%%%%%%%%%%%%%%%%%%%%%%%%%%%%%%%
\listoftables
%%%%%%%%%%%%%%%%%%%%%%%%%%%%%%%%%%%%%%%%%%%%%%%%%%%%%%%%%%%%%%%%%%%%%%%%%%%%%%%%
% Tabellen Verzeichnis ENDE
%%%%%%%%%%%%%%%%%%%%%%%%%%%%%%%%%%%%%%%%%%%%%%%%%%%%%%%%%%%%%%%%%%%%%%%%%%%%%%%%

%%%%%%%%%%%%%%%%%%%%%%%%%%%%%%%%%%%%%%%%%%%%%%%%%%%%%%%%%%%%%%%%%%%%%%%%%%%%%%%%
% Abbildungsverzeichnis START
%%%%%%%%%%%%%%%%%%%%%%%%%%%%%%%%%%%%%%%%%%%%%%%%%%%%%%%%%%%%%%%%%%%%%%%%%%%%%%%%
\listoffigures
%%%%%%%%%%%%%%%%%%%%%%%%%%%%%%%%%%%%%%%%%%%%%%%%%%%%%%%%%%%%%%%%%%%%%%%%%%%%%%%%
% Abbildungsverzeichnis ENDE
%%%%%%%%%%%%%%%%%%%%%%%%%%%%%%%%%%%%%%%%%%%%%%%%%%%%%%%%%%%%%%%%%%%%%%%%%%%%%%%%

%%%%%%%%%%%%%%%%%%%%%%%%%%%%%%%%%%%%%%%%%%%%%%%%%%%%%%%%%%%%%%%%%%%%%%%%%%%%%%%%
% Bilbiographie START
%%%%%%%%%%%%%%%%%%%%%%%%%%%%%%%%%%%%%%%%%%%%%%%%%%%%%%%%%%%%%%%%%%%%%%%%%%%%%%%%
\nocite{*}
\bibliography{Bibtex/VerzeichnisBuecher}
\addcontentsline{toc}{chapter}{Literaturverzeichnis}
\newpage
%%%%%%%%%%%%%%%%%%%%%%%%%%%%%%%%%%%%%%%%%%%%%%%%%%%%%%%%%%%%%%%%%%%%%%%%%%%%%%%%
% Bilbiographie ENDE
%%%%%%%%%%%%%%%%%%%%%%%%%%%%%%%%%%%%%%%%%%%%%%%%%%%%%%%%%%%%%%%%%%%%%%%%%%%%%%%%

%%%%%%%%%%%%%%%%%%%%%%%%%%%%%%%%%%%%%%%%%%%%%%%%%%%%%%%%%%%%%%%%%%%%%%%%%%%%%%%%
% UML START
%%%%%%%%%%%%%%%%%%%%%%%%%%%%%%%%%%%%%%%%%%%%%%%%%%%%%%%%%%%%%%%%%%%%%%%%%%%%%%%%
\chapter*{UML Diagramme}
\addcontentsline{toc}{chapter}{UML Diagramme}
%%%%%%%%%%%%%%%%%%%%%%%%%%%%%%%%%%%%%%%%%%%%%%%%%%%%%%%%%%%%%%%%%%%%%%%%%%%%%%%%
% UML ENDE
%%%%%%%%%%%%%%%%%%%%%%%%%%%%%%%%%%%%%%%%%%%%%%%%%%%%%%%%%%%%%%%%%%%%%%%%%%%%%%%%
\end{document}
